\chapter{Hardware Evaluation\index{Evaluation!Hardware}}
\label{cha:HardwareEvaluation}
Im Verlauf der Arbeit standen nacheinander zwei verschiedene Server mit unterschiedlich leistungsf�higen Komponenten f�r Testzwecke zur Verf�gung (im Folgenden mit "`Alpha Lab"' und "`Beta Lab"' bezeichnet). Auf beiden Rechnern wurde Red Hat 7.3 und VMware Workstation Version 3.2 installiert. Im Folgenden werden die Hardwarekonfigurationen der beiden Systeme und das Verhalten der Systeme w�hrend dem Einrichten und den ersten Funktionstests geschildert, um einen Eindruck �ber die Anforderungen zu vermitteln.
%----------
\section{Alpha Lab}
\label{sec:AlphaLab}
Das zur Verf�gung stehende Hostsystem hat folgende Hardwaredaten:
\begin{itemize}
\item 3.2 GB Festplatte
\item Hauptspeicher 128 MB
\item Intel Pentium II, 266 MHz
\item ATI Mach64, 4 MB RAM
\item Microsoft Maus (PS/2), zwei Kn�pfe
\item SVGA f�higer Bildschirm
\item Standard 105 Tasten Keyboard
\end{itemize}
Es konnten maximal zwei virtuelle Systeme gleichzeitig gestartet werden. Pro Systemen stand maximal 32 MB Hauptspeicher zur Verf�gung. Die Anzahl parallel aufgesetzter Systeme wurde auch durch die zur Verf�gung stehende Festplattenkapazit�t stark eingeschr�nkt. Pro Server sollte mindestens 1 GB zur Verf�gung stehen. Davon wird 1 GB bereits vom Hostsystem selbst verwendet. Die Systemreaktionen der virtuellen Systeme waren sehr tr�ge. Die Konfigurationen aus den Konsolen der simulierten Server waren noch gut m�glich. Der Bildaufbau dauerte teilweise ein bis zwei Sekunden. Bei Webzugriffen von einem beliebigen System im Uebungsnetz war nicht ersichtlich, dass es sich nur um einen simulierten Server handelte. Um zwei Webserver, ohne zus�tzliche Dienste f�r eine �bung zu simulieren, w�rden diese Hardwarevoraussetzungen reichen. Das System w�re aber sehr verletzlich gegn�ber einer \emph{Denial of Service Attacke}.
%------------------
\section{Beta Lab}
\label{sec:BetaLab}
Das zur Verf�gung stehende Hostsystem hat folgende Hardwaredaten:
\begin{itemize}
\item 6.4 GB Festplatte
\item Hauptspeicher 512 MB SDRAM
\item Intel Pentium III, 550 MHz
\item ATI Mach64, 16 MB RAM
\item Microsoft Maus (PS/2), zwei Kn�pfe
\item SVGA f�higer Bildschirm
\item Standard 105 Tasten Keyboard
\end{itemize}
Das starten von drei virtuellen Systemen gleichzeitig stellt kein Problem dar. Jedem virtuellen System stehen 128 MB Hauptspeicher zur Verf�gung. Das Arbeiten mit den virtuellen Systemen war ohne gr�ssere, bemerkbare Geschwindigekitseinbussen m�glich.
%------------------
\section{Erkenntnis}
Die Systemvoraussetzungen des Beta Labs waren gegn�ber dem Alpha Lab wesentlich besser. Die Gr�sse der Harddisk reicht zwar aus, l�sst aber f�r die Serverinstallationen nicht allzuviel Spielraum f�r das Installieren von zus�tzlichen Paketen. Es stehen pro System (dem Host System inklusive) 1.6 GB Plattenspeicher zur Verf�gung. Um m�glichen Speicherplatzproblemen vorzubeugen, sollte das Host System mindestens 8 GB Festplattenspeicher besitzen, damit den drei virtuellen Systemen je 2 GB zur Verf�gung stehen. Ansonsten gen�gen die Hardwarevoraussetzungen des Beta Labs den �bungsanforderungen.



