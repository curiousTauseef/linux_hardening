%Bitte beim eintragen alphabetisch ordnen
\chapter{Glossar}
\label{cha:Glossar}

\begin{description}
\item[\Large{A}]
\item[ARP] Address Resolution Protocol. Das Protokoll verbindet die IP-Adresse mit der physikalischen MAC-Adresse der jeweiligen Ethernet-Karte.
\item[Authentifizierung] Bezeugen der Echtheit
\item[Authentisierung] Beglaubigung, Rechtsg�ltigmachung

\item[\Large{D}]
\item[Dienst] Ein Programm, dass Funktionen oder Ressourcen zur Verf�gung stellt (Beispielsweise \textit{sshd}).

\item[\Large{H}]
\item[H�rten eines Systems] Vorgehen, um ein System sicherer zu machen.
\item[Host System] System auf welchem VMware installiert ist.

\item[\Large{M}]
\item[MAC] Media Access Control. Dieser Layer bietet die M�glichkeit des individuellen Zugriffs bez�glich der verwendeten Netzarchitektur.

\item[\Large{N}]
\item[Non-Repudiation] Eindeutigkeit des Ursprungs.

\item[\Large{O}]
\item[OS Hardening] Prozess um die Sicherheit eines IT-Systems zu erh�hen.

\item[\Large{P}]
\item[PAM] Pluggable Authentication Module. PAM dient der einheitlichen Authentifizierung der Benutzer durch verschieden Programme.
\item[partitionieren] Die Festplatte in verschiedene Bereiche aufteilen, die untereinander unabh�ngig sind.

\item[\Large{V}]
\item[Virtuelles System] Ein innerhalb von VMware installiertes System, das einen kompletten Rechner simuliert. Die Hardwarekonfiguration des Rechners sind simulierte VMware Komponenten.
\end{description}
