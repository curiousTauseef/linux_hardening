\chapter{Schlussfolgerungen}
\label{cha:Schlussfolgerungen}

\section{Fazit}
Es ist schwierig, H�rten eines Systems einzugrenzen und festzulegen, was alles dazugeh�rt. Diese Arbeit zeigt verschiedene Aspekte auf, die beim H�rten eines Systems beachtet werden sollten. Welche Aspekte dabei wichtiger bzw. weniger wichtig sind, h�ngt von der Verwendung des Systems ab und muss von Fall zu Fall neu entschieden werden. Trotzdem ist es gelungen, einen �berblick zu geben, der erlaubt, die Prinzipien zu verstehen und auch das Gef�hl daf�r gibt, was mit H�rten alles erreicht werden kann. Da es immer neue Programme gibt, wird es auch immer neue Anforderungen an die Sicherheit geben, wenn man aber verstanden hat, worauf man achten sollte, wird man keine grosse Schwierigkeiten haben, neue Sicherheitsanforderungen zu verstehen und umzusetzen.

Die erarbeitete �bung bietet eine M�glichkeit, das gewonnene Wissen in der Praxis umzusetzen. Dabei wurde darauf geachtet, dass die �bung nicht aus Fragen besteht, die einfach aus einem Text abgeschrieben und somit schnell beantwortet werden k�nnen, sondern, Fragen zu stellen, zu deren Beantwortung das Verst�ndnis f�r das entsprechende Gebiet vorhanden sein muss. Damit wurde versucht, die Aufgaben m�glichst spannend zu gestalten. Sie geben dar�ber hinaus Aufschluss �ber Sicherheitsrisiken, deren Vorhandensein kaum jemandem bewusst ist, sich aber dennoch viele aussetzen. Damit soll ein \glqq Aha-Effekt\grqq\ entstehen.

\section{Ausblick}
Die in dieser Arbeit erzielten Resultate entsprechen nur einem Ausschnitt aus dem Gebiet der Sicherheit. Wie immer, ist damit nur die \glqq halbe Wahrheit\grqq\ gesprochen bzw. geschrieben worden. Es gibt noch sehr viele Richtungen, aus der die Thematik Sicherheit angegangen werden kann. Wollte man die \glqq ganze Wahrheit\grqq\ wissen, w�re dies schier ein Ding der Unm�glichkeit. Wobei dazu noch zu sagen ist, dass dies auch nicht unbedingt n�tig ist. Sicherheit ist ein dehnbarer Begriff, der je nach Anwendung, je nach pers�nlicher Ansicht und je nach Risiko unterschiedlich ausgelegt werden kann. Nebst dem H�rten eines Systems gibt es noch weitere Gebiete, die ebenso wichtig sind. Namentlich sind das Intrusion Prevention, Intrusion Detection, Firewalling und andere. Jedes dieser Gebiete w�rde f�r sich eine eigene Arbeit ergeben. Damit soll nur gezeigt werden, dass diese Arbeit einen Ausschnitt der grossen Welt der Sicherheit bietet und nicht als Vorlage genommen werden kann, ein System komplett sicher zu machen.

Ebenso wie das H�rten eines Systems nur ein Ausschnitt ist, sind die in dieser Arbeit beschriebenen Vorgehensweisen und Vorschl�ge nur ein Ausschnitt aus allen M�glichkeiten, die das H�rten bietet. Die Auswahl erfolgte, nach Studium aller gefundenen M�glichkeiten, nach der Wichtigkeit aus Sicht der Autoren. Man k�nnte sich durchaus vorstellen, die Arbeit in diesem Sinne zu erweitern und noch mehr M�glichkeiten des H�rtens aufzuzeigen. Vor allem aber ist es wichtig, sich selbst auf einem aktuellen Stand zu halten, da sich im Bereich der Sicherheit sehr viel in kurzer Zeit �ndert.

Die �bung l�sst sich fast beliebig ausbauen. Das Ziel war bei der Erstellung, m�gliche Fragen und deren Antworten bereitzustellen. Aufbauend auf diesen Fragen k�nnen neue, andere Fragen entstehen, die je nach Ziel und Anforderungen der �bung verschieden ausfallen k�nnen.
